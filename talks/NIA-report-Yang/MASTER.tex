%%%%%%%%%%%%%%%%%%%%%%%%%%%%%%%%%%%%%%%%%%%%%%%%%%%%%%%%%%%%%%%%%%%%%%%%%%
% This is the master file with the symbols used for most, hopefully all, 
% of my publications in the areas of Markov chains, model checking, Petri
% nets, etc.
%%%%%%%%%%%%%%%%%%%%%%%%%%%%%%%%%%%%%%%%%%%%%%%%%%%%%%%%%%%%%%%%%%%%%%%%%%

\usepackage{ifthen}
\usepackage{epsfig}
\usepackage{graphics}
\usepackage{color}
\usepackage{amsmath}
\usepackage{amssymb}
\usepackage{latexsym}
\usepackage{graphics}
\usepackage{float}                % to create custom float environments
\usepackage[bf]{caption}          % boldface captions in float env.
%\usepackage[bf,small]{caption}    % boldface small captions in float env.


%%%%%%%%%%%%%%%%%%%%%%%%%%%%%%%%%%%%%%%%%%%%%%%%%%%%%%%%%%%%%%%%%%%%%%%%%%
% SMART

\newcommand{\smartmeaning}{Stochastic Model checking Analyzer for Reliability and Timing}
%%%% use within "{}", as in "{\smartmeaning}", for correct spacing

%\newcommand{\Smart}{{\textsc{Smart}}}
\newcommand{\smart}{{S%
    \kern-.11em\raise.39ex\hbox{\footnotesize{M}}%
    \kern-.22em\raise.0ex\hbox{A}%
    \kern-.21em\raise.39ex\hbox{\footnotesize{R}}%
    \kern-.16em T}}
\renewcommand{\smart}{{\sc{S%
    \kern-.11em\raise.39ex\hbox{m}%
    \kern-.22em\raise.0ex\hbox{A}%
    \kern-.21em\raise.39ex\hbox{r}%
    \kern-.16em T}}}
%%%% use within "{}", as in "{\smart}", for correct spacing

\newcommand{\NEW}{\mbox{\PSSCALE{NEW}{0.3}}} 

%%%%%%%%%%%%%%%%%%%%%%%%%%%%%%%%%%%%%%%%%%%%%%%%%%%%%%%%%%%%%%%%%%%%%%%%%%
% Colors

\definecolor{mediumblue}{rgb}{0.5,0.5,1.0}
\definecolor{orange}{rgb}{1.0,0.6,0.2}
\definecolor{mediumyellow}{rgb}{1.0,1.0,0.5}
\definecolor{lightyellow}{rgb}{1.0,1.0,0.6}
\definecolor{mediumgreen}{rgb}{0.5,1.0,0.5}
\definecolor{darkgreen}{rgb}{0.0,0.5,0.0}
\definecolor{lightgreen}{rgb}{0.7,1.0,0.7}
\definecolor{lightgrey}{rgb}{0.8,0.8,0.8}
\definecolor{mediumgrey}{rgb}{0.5,0.5,0.5}
\definecolor{darkgrey}{rgb}{0.3,0.3,0.3}
\newcommand{\MAGENTA}[1]{\textcolor{magenta}{#1}}
\newcommand{\CYAN}[1]{\textcolor{cyan}{#1}}
\newcommand{\ORANGE}[1]{\textcolor{orange}{#1}}
\newcommand{\BLUE}[1]{\textcolor{blue}{#1}}
\newcommand{\MEDIUMBLUE}[1]{\textcolor{mediumblue}{#1}}
\newcommand{\RED}[1]{\textcolor{red}{#1}}
\newcommand{\YELLOW}[1]{\textcolor{yellow}{#1}}
\newcommand{\MEDIUMYELLOW}[1]{\textcolor{mediumyellow}{#1}}
\newcommand{\LIGHTYELLOW}[1]{\textcolor{lightyellow}{#1}}
\newcommand{\GREEN}[1]{\textcolor{green}{#1}}
\newcommand{\MEDIUMGREEN}[1]{\textcolor{mediumgreen}{#1}}
\newcommand{\DARKGREEN}[1]{\textcolor{darkgreen}{#1}}
\newcommand{\LIGHTGREEN}[1]{\textcolor{lightgreen}{#1}}
\newcommand{\BLACK}[1]{\textcolor{black}{#1}}
\newcommand{\WHITE}[1]{\textcolor{white}{#1}}
\newcommand{\LIGHTGREY}[1]{\textcolor{lightgrey}{#1}}
\newcommand{\MEDIUMGREY}[1]{\textcolor{mediumgrey}{#1}}
\newcommand{\DARKGREY}[1]{\textcolor{darkgrey}{#1}}

%%%%%%%%%%%%%%%%%%%%%%%%%%%%%%%%%%%%%%%%%%%%%%%%%%%%%%%%%%%%%%%%%%%%%%%%%%
% Miscellanea

\newcommand{\IGNORE}[1]{}                        % Ignore the argument
\newcommand{\TBD}{{\textcolor{red}{\bf TBD}}}    % To Be Done 
\newcommand{\MSG}[1]{{\color{blue} $<\!\!<\!\!<$ #1 $>\!\!>\!\!>$}} % messages

\newcommand{\SUPST}{\mbox{$^{\mathrm{st}}$}}     % to put a ^st in text
\newcommand{\SUPND}{\mbox{$^{\mathrm{nd}}$}}     % to put a ^nd in text
\newcommand{\SUPRD}{\mbox{$^{\mathrm{rd}}$}}     % to put a ^rd in text
\newcommand{\SUPTH}{\mbox{$^{\mathrm{th}}$}}     % to put a ^th in text
%\newcommand{\id}[1]{\mbox{$\mathit{#1}$}}        % identifier
\newcommand{\id}[1]{\mathit{#1}}                 % identifier
\newcommand{\COMMA}{\mbox{,}}                    % comma in math mode numbers

%%%%%%%%%%%%%%%%%%%%%%%%%%%%%%%%%%%%%%%%%%%%%%%%%%%%%%%%%%%%%%%%%%%%%%%%%%
% Floating environments

\floatstyle{plain}                     % or boxed, or ruled
\newfloat{MyFloatEnv}{thb}{}[section]  % or chapter, or ..., [] arg is optional
\floatname{MyFloatEnv}{Caption Name}   % text created by \caption
%%% to use, type:
%%% \begin{MyFloatEnv} ... \caption{...} \label{...} \end{MyFloatEnv}

%%%%%%%%%%%%%%%%%%%%%%%%%%%%%%%%%%%%%%%%%%%%%%%%%%%%%%%%%%%%%%%%%%%%%%%%%%
% Probability

\newcommand{\Prob}[1]{\mbox{Pr}\left\{#1\right\}}      % probability
\newcommand{\E}[1]{\mbox{E}\left[#1\right]}          % expectation
\newcommand{\Var}[1]{\mbox{Var}\left(#1\right)}      % variance
\newcommand{\Const}{\mbox{Const}}
\newcommand{\Bernoulli}{\mbox{Bernoulli}}
\newcommand{\Binomial}{\mbox{Binomial}}
\newcommand{\Geom}{\mbox{Geom}}
\newcommand{\ModGeom}{\mbox{ModGeom}}
\newcommand{\Equilikely}{\mbox{Equilikely}}
\newcommand{\Discrete}{\mbox{Discrete}}
\newcommand{\Poisson}{\mbox{Poisson}}
\newcommand{\Expo}{\mbox{Expo}}
\newcommand{\Erlang}{\mbox{Erlang}}
\newcommand{\Hypo}{\mbox{Hypo}}
\newcommand{\Hyper}{\mbox{Hyper}}
\newcommand{\Unif}{\mbox{Unif}}
\newcommand{\Normal}{\mbox{Normal}}

%%%%%%%%%%%%%%%%%%%%%%%%%%%%%%%%%%%%%%%%%%%%%%%%%%%%%%%%%%%%%%%%%%%%%%%%%%
% Numeric sets

\newcommand{\Reals}{\mathbb{R}}      % real numbers
\newcommand{\Naturals}{\mathbb{N}}   % natural numbers
\newcommand{\Integers}{\mathbb{Z}}   % integer numbers
\newcommand{\Rationals}{\mathbb{Q}}  % rational numbers
\newcommand{\Complexes}{\mathbb{C}}  % complex numbers

%%%%%%%%%%%%%%%%%%%%%%%%%%%%%%%%%%%%%%%%%%%%%%%%%%%%%%%%%%%%%%%%%%%%%%%%%%
% Mathematical symbols

\newcommand{\THEN}{\Rightarrow}                     % implication
\newcommand{\IFF}{\Leftrightarrow}                  % biimplication
\newcommand{\IDENT}{\equiv}                         % identicaly equal
\newcommand{\EQDEF}{=_{\mathrm{df}}} 
\newcommand{\DELTA}[1]{\delta_{#1}}                 % 1 if true, 0 if false
\newcommand{\DET}[1]{\mathrm{det}({#1})}            % determinant
\newcommand{\RANK}[1]{\mathrm{rank}({#1})}          % rank
\newcommand{\DIAG}[1]{\mathrm{diag}({#1})}          % diagonal matrix
\newcommand{\FLOAT}[2]{\mbox{#1$\times$10$^{\mbox{\scriptsize{#2}}}$}}
                                                    % expo notation
\newcommand{\kplus}{\bigoplus}                      % Kronecker sum
\newcommand{\ktimes}{\bigotimes}                    % Kronecker product
\newcommand{\bigCH}[2]{\left(\begin{array}{@{}c@{}}{#1}\\[-0.4ex]
                           {#2}\end{array}\right)}  % N choose M
\newcommand{\CH}[2]{\left(\begin{array}{@{}c@{}}~\\[-4.5ex]{#1}\\[-1.4ex]
                           {#2}\\[-0.6ex]\end{array}\right)}  % N choose M


%%%%%%%%%%%%%%%%%%%%%%%%%%%%%%%%%%%%%%%%%%%%%%%%%%%%%%%%%%%%%%%%%%%%%%%%%%
% Vectors, matrices, and sets

\newcommand{\vect}[1]{\mathbf{#1}}               % does not bold lowercase greek
\newcommand{\gvect}[1]{\mbox{\boldmath{${#1}$}}} % wrong size for scripts
%\newcommand{\gvect}[1]{\mathbf{#1}} % wrong size for scripts
\newcommand{\matr}[1]{\mathbf{#1}}               % matrices
\newcommand{\Set}[1]{\mathcal{#1}}

%%%%%%%%%%%%%%%%%%%%%%%%%%%%%%%%%%%%%%%%%%%%%%%%%%%%%%%%%%%%%%%%%%%%%%%%%%
% Petri nets

\newcommand{\mk}[1]{{\framebox{#1}}}     % marking
\newcommand{\goesto}[1]{{\stackrel{#1}{\rightharpoondown}}} % firing

\newcommand{\PRE}{\succ}
\newcommand{\POST}{\succ\!\!\!\succ}

%%%%%%%%%%%%%%%%%%%%%%%%%%%%%%%%%%%%%%%%%%%%%%%%%%%%%%%%%%%%%%%%%%%%%%%%%%
% Formal languages

\newcommand{\goes}{\mbox{$\rightarrow$}}     % grammar rule
% derivation
\newcommand{\derive}{\Longrightarrow}
\newcommand{\deriveU}[1]{\stackrel{#1}{\Longrightarrow}}
\newcommand{\deriveL}[1]{{\renewcommand{\arraystretch}{0.4} \begin{array}{c}
   {~}\\{\Longrightarrow}\\{\scriptstyle {#1}}\end{array}}}
\newcommand{\deriveUL}[2]{{\renewcommand{\arraystretch}{0.4} \begin{array}{c}
   {\scriptstyle {#1}}\\{\Longrightarrow}\\{\scriptstyle {#2}}\end{array}}}
% computation
\newcommand{\compute}{\vdash}
\newcommand{\computeU}[1]{\stackrel{#1}{\vdash}}
\newcommand{\computeL}[1]{{\renewcommand{\arraystretch}{0.4} \begin{array}{c}
   {~}\\{\vdash}\\{\scriptstyle {#1}}\end{array}}}
\newcommand{\computeUL}[2]{{\renewcommand{\arraystretch}{0.4} \begin{array}{c}
   {\scriptstyle {#1}}\\{\vdash}\\{\scriptstyle {#2}}\end{array}}}
\newcommand{\group}[1]{\framebox{${#1}$}}
\newcommand{\halts}{\searrow}
\newcommand{\diverges}{\nearrow}


%%%%%%%%%%%%%%%%%%%%%%%%%%%%%%%%%%%%%%%%%%%%%%%%%%%%%%%%%%%%%%%%%%%%%%%%%%
% Decision diagrams

\newcommand{\Ptr}[2]{\langle{#1}{|}{#2}\rangle}
  % node in a K-variable MDD
\newcommand{\PPtr}[2]{\langle\!\!\,\langle{#1}{|}{#2}\rangle\!\!\,\rangle}
  % node in a 2K-variable MDD
\newcommand{\Top}{\id{Top}} \newcommand{\Bot}{\id{Bot}} 
\newcommand{\EVMDD}{{EV$^+$\hspace*{-0.25ex}MDD}}


%%%%%%%%%%%%%%%%%%%%%%%%%%%%%%%%%%%%%%%%%%%%%%%%%%%%%%%%%%%%%%%%%%%%%%%%%%
% Temporal logic operators

%%%% \newcommand{\true}{\top}
%%%% \necommand{\false}{\bot}
\newcommand{\tl}[1]{{\bf\textsf{#1}}}
\newcommand{\tF}{\tl{F}}            % finally operator
\newcommand{\tG}{\tl{G}}            % globally operator
\newcommand{\tX}{\tl{X}}            % next operator
\newcommand{\tU}{\tl{U}}            % until operator
\newcommand{\tR}{\tl{R}}            % releases operator
\newcommand{\tA}{\tl{A}}            % for all paths operator
\newcommand{\tE}{\tl{E}}            % exists a path operator
\newcommand{\tEF}{\tl{EF}}          % CTL EF operator
\newcommand{\tEG}{\tl{EG}}          % CTL EG operator
\newcommand{\tEX}{\tl{EX}}          % CTL EX operator
\newcommand{\tEU}{\tl{EU}}          % CTL EU operator
\newcommand{\tER}{\tl{ER}}          % CTL ER operator
\newcommand{\tAF}{\tl{AF}}          % CTL AF operator
\newcommand{\tAG}{\tl{AG}}          % CTL AG operator
\newcommand{\tAX}{\tl{AX}}          % CTL AX operator
\newcommand{\tAU}{\tl{AU}}          % CTL AU operator
\newcommand{\tAR}{\tl{AR}}          % CTL AR operator


%%%%%%%%%%%%%%%%%%%%%%%%%%%%%%%%%%%%%%%%%%%%%%%%%%%%%%%%%%%%%%%%%%%%%%%%%%
% Special predefined symbols

\newcommand{\I}{\matr{I}}             % the identity matrix
\newcommand{\Shuffle}{\matr{S}}       % the shuffle matrix
\renewcommand{\P}{\matr{P}}           % transition probability matrix
\newcommand{\N}{\matr{N}}             % fundamental matrix
\newcommand{\Q}{\matr{Q}}             % infinitesimal generator
\newcommand{\R}{\matr{R}}             % transition rate matrix
\newcommand{\G}{\matr{G}}             % matrix for MG1
\newcommand{\B}{\matr{B}}             % matrix (Backward)
\renewcommand{\L}{\matr{L}}           % matrix (Local)
\newcommand{\F}{\matr{F}}             % matrix (Forward)
\newcommand{\W}{\matr{W}}             % local real matrix for Kronecker
\newcommand{\T}{\matr{T}}             % local boolean matrix for Kronecker
\newcommand{\vphi}{\gvect{\phi}}      % state entrance rate vector
\newcommand{\vpi}{\gvect{\pi}}        % probability vector
\newcommand{\vgamma}{\gvect{\gamma}}  % probability vector
\newcommand{\vp}{\vect{p}}            % probability vector
\newcommand{\vq}{\vect{q}}            % probability vector
\newcommand{\vh}{\vect{h}}            % holding time vector 
\newcommand{\vn}{\vect{n}}            % sojourn time in states (discrete time)
\newcommand{\vsigma}{\gvect{\sigma}}  % sojourn time in states (continuous time)
\newcommand{\vdelta}{\gvect{\delta}}  % 
\newcommand{\vtau}{\gvect{\tau}}      % RFT vector
\newcommand{\vrho}{\gvect{\rho}}      % reward vector 
\newcommand{\vepsilon}{\gvect{\epsilon}}  % error vector 
\newcommand{\initstate}{\vs^{init}}   % initial (starting) state
\newcommand{\initstateset}{\sset^{init}}  % initial (starting) set of states
\newcommand{\vm}{\vect{m}}            % generic state
\newcommand{\vs}{\vect{s}}            % generic state
\newcommand{\vt}{\vect{t}}            % vector of remaining firing times
\newcommand{\vi}{\vect{i}}            % generic (from) state
\newcommand{\vj}{\vect{j}}            % generic (to) state
\newcommand{\BPi}{\matr{\Pi}}         % DTMC matrix
\newcommand{\BPhi}{\matr{\Phi}}       % matrix of firing frequencies
\newcommand{\0}{\matr{0}}             % a matrix or vector of zeros
\newcommand{\1}{\matr{1}}             % a matrix or vector of ones
%\newcommand{\e}[1]{\vect{e}_{#1}}     % a vector of zeros with a one in position #1
\newcommand{\sset}{\Set{S}}           % reachability sets (global, local)
\newcommand{\negsset}{\overline{\Set{S}}}  % complement of gloabl reach. set
\newcommand{\vanset}{\Set{V}}         % vanishing set
\newcommand{\tanset}{\Set{T}}         % tangible set
\newcommand{\absorbingset}{\Set{D}}   % absorbing (deadlocked) states
\newcommand{\transientset}{\sset^0}   % transient states
\newcommand{\eset}{\Set{E}}           % event set
\newcommand{\tset}{\Set{T}}           % transition set (PN)
\newcommand{\kset}{\Set{K}}           % set of levels
\newcommand{\lset}{\Set{L}}           % set of local states
\newcommand{\pset}{\Set{P}}           % place set
\newcommand{\nset}{\Set{N}}           % next state function (new set of states)
\newcommand{\uset}{\Set{U}}           % unexplored state set
\newcommand{\oset}{\Set{O}}           % old state set
\newcommand{\fset}{\Set{F}}           % set
\newcommand{\mset}{\Set{M}}           % set
\newcommand{\iset}{\Set{I}}           % set of from states
\newcommand{\jset}{\Set{J}}           % set of to states
\newcommand{\aset}{\Set{A}}           % set above
\newcommand{\bset}{\Set{B}}           % set below
\newcommand{\dset}{\Set{D}}           % set
\newcommand{\pot}[1]{\widehat{#1}}    % potential (as opposed to actual)
\newcommand{\potsset}{\pot{\sset}}    % potential state space
\newcommand{\potR}{\pot{\R}}          % potential transition rate matrix
\newcommand{\potQ}{\pot{\Q}}          % potential infinitesimal generator matrix
\newcommand{\potN}{\pot{\N}}          % potential boolean transition matrix
\newcommand{\potpi}{\pot{\vpi}}       % potential probability vector
\newcommand{\potn}{\pot{n}}           % number of potential states
\newcommand{\lex}{\psi}               % mapping for actual states
\newcommand{\potlex}{\pot{\lex}}      % mapping for potential states

%%%%%%%%%%%%%%%%%%%%%%%%%%%%%%%%%%%%%%%%%%%%%%%%%%%%%%%%%%%%%%%%%%%%%%%%%%
% Generic predefined symbols

\newcommand{\va}{\vect{a}}            % vector
\newcommand{\vb}{\vect{b}}            % vector
\newcommand{\vc}{\vect{c}}            % vector
\newcommand{\vd}{\vect{d}}            % vector
\newcommand{\vf}{\vect{f}}            % vector
\newcommand{\vg}{\vect{g}}            % vector
\newcommand{\vv}{\vect{v}}            % vector
\newcommand{\vx}{\vect{x}}            % vector
\newcommand{\vy}{\vect{y}}            % vector
\newcommand{\vz}{\vect{z}}            % vector
\newcommand{\A}{\matr{A}}             % matrix
\newcommand{\C}{\matr{C}}             % matrix
\newcommand{\M}{\matr{M}}             % matrix
\newcommand{\D}{\matr{D}}             % matrix
\newcommand{\U}{\matr{U}}             % matrix 
\newcommand{\cset}{\Set{C}}           % set
\newcommand{\qset}{\Set{Q}}           % generic set
\newcommand{\rset}{\Set{R}}           % set
\newcommand{\xset}{\Set{X}}           % set
\newcommand{\yset}{\Set{Y}}           % set


%%%%%%%%%%%%%%%%%%%%%%%%%%%%%%%%%%%%%%%%%%%%%%%%%%%%%%%%%%%%%%%%%%%%%%%%%%
% BNF notation

\newcommand{\DEF}[1]{\begin{description}\item \tt {#1}\end{description}}
\newcommand{\KEY}[1]{\mbox{\sl{#1}}}
\newcommand{\MetaIs}{\mbox{{\large{$\Rightarrow$}}}}
\newcommand{\MetaOr}{{\hspace{2mm}{\large{$|$}}\hspace{2mm}}}
\newcommand{\BNFREM}{\medskip \noindent{\bf Remark}~}


%%%%%%%%%%%%%%%%%%%%%%%%%%%%%%%%%%%%%%%%%%%%%%%%%%%%%%%%%%%%%%%%%%%%%%%%%%
% Theorem-like commands

\newtheorem{mynote}{Note}[section]
  \newcommand{\Note}[1]{\begin{mynote}{\rm {#1}}\end{mynote}}%
\newtheorem{mytheorem}{Theorem}[section]
  \newcommand{\Theorem}[1]{\begin{mytheorem}{\rm {#1}}\end{mytheorem}}%
\newtheorem{mycorollary}{Corollary}[section]
  \newcommand{\Corollary}[1]{\begin{mycorollary}{\rm {#1}}\end{mycorollary}}%
\newtheorem{myremark}{Remark}[section]
  \newcommand{\Remark}[1]{\begin{myremark}{\rm {#1}}\end{myremark}}%
\newtheorem{mylemma}{Lemma}[section]
  \newcommand{\Lemma}[1]{\begin{mylemma}{\rm {#1}}\end{mylemma}}%
\newtheorem{mydefinition}{Definition}[section]
  \newcommand{\Definition}[1]{\begin{mydefinition}{\rm {#1}}\end{mydefinition}}%
\newtheorem{myexample}{Example}[section]
  \newcommand{\Example}[1]{\begin{myexample}{\rm {#1}}\end{myexample}}%
\newtheorem{myalgorithm}{Algorithm}[section]
  \newcommand{\Algorithm}[1]{\begin{myalgorithm}{\rm {#1}}\end{myalgorithm}}%

\newcommand{\Proof}[1]{\par\noindent{\bf Proof.}~{#1}$\Box$}
\newcommand{\ExampleCont}[2]{\begin{myexample}{{\bf (continues Example \ref{#1})~}\rm {#2}}\end{myexample}}%


%%%%%%%%%%%%%%%%%%%%%%%%%%%%%%%%%%%%%%%%%%%%%%%%%%%%%%%%%%%%%%%%%%%%%%%%%%
% epsf figures (inline or centered)

\newcommand{\PS}[1]{\mbox{\epsfig{file=#1.eps}} } 
\newcommand{\PSWIDE}[2]{\mbox{\epsfig{file=#1.eps, width=#2}}} 
\newcommand{\PSHIGH}[2]{\mbox{\epsfig{file=#1.eps, height=#2}}} 
\newcommand{\PSSCALE}[2]{\mbox{\epsfig{file=#1.eps, scale=#2}}}

\newcommand{\CENTERPS}[1]{\begin{center}\mbox{\epsfig{file=#1.eps}}\end{center}}
\newcommand{\FPSCENTERWIDE}[2]{\begin{center}\mbox{\epsfig{file=FIGURES/#1.eps, width=#2}}\end{center}}
\newcommand{\CENTERPSWIDE}[2]{\begin{center}\mbox{\epsfig{file=#1.eps, width=#2}}\end{center}}
\newcommand{\CENTERPSHIGH}[2]{\begin{center}\mbox{\epsfig{file=#1.eps, height=#2}}\end{center}}
\newcommand{\CENTERPSSCALE}[2]{\begin{center}\mbox{\epsfig{file=#1.eps, scale=#2}}\end{center}}

\newcommand{\INLINEFIG}[1]{\epsfclipoff\epsffile{#1.eps}}
\newcommand{\FIG}[1]{\begin{center} \mbox{\epsfclipoff\epsffile{#1.eps}} \end{center}}
\def\epsfsize#1#2{0.50#1}   % change the "0.50" zoom factor to anything you want



%%%%%%%%%%%%%%%%%%%%%%%%%%%%%%%%%%%%%%%%%%%%%%%%%%%%%%%%%%%%%%%%%%%%%%%%%%
% Cross-references

\newcommand{\EquationRef}[1]{Equation~\ref{#1}}
\newcommand{\FigureRef}[1]{Figure~\ref{#1}}
\newcommand{\TableRef}[1]{Table~\ref{#1}}
\newcommand{\SectionRef}[1]{Section~\ref{#1}}
\newcommand{\ChapterRef}[1]{Chapter~\ref{#1}}
\newcommand{\AppendixRef}[1]{Appendix~\ref{#1}}


%%%%%%%%%%%%%%%%%%%%%%%%%%%%%%%%%%%%%%%%%%%%%%%%%%%%%%%%%%%%%%%%%%%%%%%%%%
% Programming and pseudocode 

\newcommand{\Code}[1]{{\tt {#1}}}     % for SMART code
\newcommand{\Unix}[1]{{\tt {#1}}}     % for command-line stuff
\newenvironment{code}{\begin{quote}\noindent\hspace*{-3ex}\begin{minipage}{0.91\columnwidth}}{\end{minipage}\end{quote}}

\newlength{\FIRSTLISTINGSPACING} \setlength{\FIRSTLISTINGSPACING}{-1.50ex}
\newlength{\LISTINGSPACING} \setlength{\LISTINGSPACING}{-0.80ex}
\newlength{\LISTINGINDENT} \setlength{\LISTINGINDENT}{0.55em}
% use ALGORITHM for an algorithm having a title for 1st parameter and
% ALGDESCR and (PLAIN)LISTING for 2nd parameter
\newcommand{\ALGORITHM}[2]{\noindent\framebox{\begin{minipage}[t]{0.962\columnwidth}\vspace*{-0.0ex}\noindent \textsf{#1}\noindent\textsf{#2}\end{minipage}}}
% use ALGDESCR to give an english description of an algorithm
\newcommand{\ALGDESCR}[1]{\\[-1.9ex]\hrule\vspace*{0.8ex}\textsf{#1}\\[-1.5ex]\hrule}
% use LISTING for pseudocode with line numbers (the 1st parameter is the
%initial line count, use 0 unless it is just a portion of an algorithm)
% use PLAINLISTING for a portion of pseudocode without line numbers
\newcommand{\LOCAL}{\ref{}}   % should never be used outside a (PLAIN)LISTING
\newcommand{\PLAINLISTING}[1]{\vspace*{-1.5ex}
  \renewcommand{\LOCAL}{\item[] local~} % local variable
  \newcommand{\GLOBAL}{\item[] global~} % local variable
  \begin{description}{\sf \setlength{\itemsep}{\LISTINGSPACING}
  {#1}}\end{description}
} 
\newcommand{\LISTING}[2]{\vspace*{\FIRSTLISTINGSPACING}
  \renewcommand{\LOCAL}{\item[] \hspace*{-1.5em}local~} % local variable
  \newcommand{\GLOBAL}{\item[] \hspace*{-1.5em}global~} % global variable
  \begin{enumerate}{\sf \setlength{\itemsep}{\LISTINGSPACING}
  \setcounter{enumi}{#1}{#2}}\end{enumerate}%
}
\newcommand{\LL}{\item\hspace*{0\LISTINGINDENT}}
\newcommand{\LLx}{\item\hspace*{1\LISTINGINDENT}}
\newcommand{\LLxx}{\item\hspace*{2\LISTINGINDENT}}
\newcommand{\LLxxx}{\item\hspace*{3\LISTINGINDENT}}
\newcommand{\LLxxxx}{\item\hspace*{4\LISTINGINDENT}}
\newcommand{\LLxxxxx}{\item\hspace*{5\LISTINGINDENT}}
\newcommand{\LLxxxxxx}{\item\hspace*{6\LISTINGINDENT}}
\newcommand{\LLxxxxxxx}{\item\hspace*{7\LISTINGINDENT}}
\newcommand{\LLxxxxxxxx}{\item\hspace*{8\LISTINGINDENT}}
\newcommand{\LLxxxxxxxxx}{\item\hspace*{9\LISTINGINDENT}}
\newcommand{\LLxxxxxxxxxx}{\item\hspace*{10\LISTINGINDENT}}
\newcommand{\LLxxxxxxxxxxx}{\item\hspace*{11\LISTINGINDENT}}
\newcommand{\LLxxxxxxxxxxxx}{\item\hspace*{12\LISTINGINDENT}}
\newcommand{\LLxxxxxxxxxxxxx}{\item\hspace*{13\LISTINGINDENT}}
\newcommand{\NULL}{\textup{\textsf{null}}}             % null pointer/value
\newcommand{\ASSIGN}{\mbox{$~\leftarrow~$}}            % assignment
\newcommand{\EXCHANGE}{\mbox{$~\leftrightarrow~$}}     % value exchange
\newcommand{\REMARK}[1]{\hfill {$\bullet$~{\sl{#1}}}}  % comments in pseudocode



