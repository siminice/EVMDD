\FOIL{Basic idea}

\LIST{
\BULL $\mathcal{D}_{init}(\vi)$ is the length of the shortest path from $\sset_{init}$ to $\vi$. $\mathcal{D}_{init}(\sset_{init})=0$
\BULL $\mathcal{D}(\vi,\vj)$ is the length of the shortest non-trival path from $\vi$ to $\vj$.
}
Let $\mathcal{LW}$ be a function $\sset_{rch} \mapsto \mathbb{N} \cup \{\infty$\} and 
$$
\mathcal{LW}(\vi) = \mathcal{D}_{init}(\vi) + \mathcal{D}(\vi, \vi)
$$
The length of the shortest witness is: $\id{Min}(\mathcal{LW}(\vi))$

\FPSCENTERWIDE{witness2}{4cm}

\LIST{
\BULL $\mathcal{D}_{init}(\vi)$: \emph{``Using Edge-Valued Decision Diagrams for Symbolic Generation of Shortest Paths''} (FMCAD'02)
    \LIST{
    \CIRC employ edge-valued decision diagrams (EVMDD) to encode the distance function to the initial state.
    \CIRC build the distance function using saturation.
    \CIRC find the shortest path from each state to the initial state.
    }

\BLUE{
\BULL $\mathcal{D}(\vi, \vi)$: ongoing.  
    \LIST{
    \CIRC employ a $2L$-level EVMDD to encode $\mathcal{D}(\vi,\vj)$, where $\vi, \vj \in \sset_{rch}$. 
    \CIRC build $\mathcal{D}(\vi,\vj)$ using saturation combining the saturation-based transitive closure computation and idea in FMCAD'02 paper. 
    }
}
}


