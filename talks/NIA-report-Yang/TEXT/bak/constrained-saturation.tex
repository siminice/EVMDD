\FOIL{Constrained saturation - Motivation}

\vfill

Constrained saturation extends the saturation algorithm by constraining state-space exploration in a given set of states.

Motivation: the computation of the CTL operator $\tE [p \tU q]$.

\FPSCENTERWIDE{ctl-ops0}{5cm}

%$$\tE[p \tU q] = \mu y.(q \lor y \lor (p \land \nset^{-1}(y)))$$
%
%Objective: to constrain the saturation-based state-space exploration within a given set of states.

\vfill

\FOIL{Constrained saturation - Motivation}

\vfill
Constrained saturation extends the saturation algorithm by constraining state-space exploration in a given set of states.

Motivation: the computation of the CTL operator $\tE [p \tU q]$.

\FPSCENTERWIDE{ctl-ops1}{5cm}

$$\tE[p \tU q] = \mu y.(q \lor y \lor (p \land \nset^{-1}(y)))$$

\vfill

\FOIL{Constrained saturation - Motivation}

\vfill
Constrained saturation extends the saturation algorithm by constraining state-space exploration in a given set of states.

Motivation: the computation of the CTL operator $\tE [p \tU q]$.

\FPSCENTERWIDE{ctl-ops2}{5cm}

$$\tE[p \tU q] = \mu y.(q \lor y \lor (p \land \nset^{-1}(y)))$$

\vfill

\FOIL{Constrained saturation - Motivation}

\vfill
Constrained saturation extends the saturation algorithm by constraining state-space exploration in a given set of states.

Motivation: the computation of the CTL operator $\tE [p \tU q]$.

\FPSCENTERWIDE{ctl-ops3}{5cm}

$$\tE[p \tU q] = \mu y.(q \lor y \lor (p \land \nset^{-1}(y)))$$

\vfill

\FOIL{Constrained saturation - Motivation}

\vfill
Constrained saturation extends the saturation algorithm by constraining state-space exploration in a given set of states.

Motivation: the computation of the CTL operator $\tE [p \tU q]$.

\FPSCENTERWIDE{ctl-ops4}{5cm}

$$\tE[p \tU q] = \mu y.(q \lor y \lor (p \land \nset^{-1}(y)))$$

\vfill

\FOIL{Constrained saturation - Motivation}

\vfill
Constrained saturation extends the saturation algorithm by constraining state-space exploration in a given set of states.

Motivation: the computation of the CTL operator $\tE [p \tU q]$.

\FPSCENTERWIDE{ctl-ops0}{5cm}

$$\tE[p \tU q] = \mu y.(q \lor y \lor (p \land \nset^{-1}(y)))$$

\vfill
